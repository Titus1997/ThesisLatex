\documentclass[../Thesis.tex]{subfiles}
 
\begin{document}
\section {Results}
..

\section {Discussion}
..

\section {Future Work}

Our software, obviously has not yet reached its potential. There is a lot of room for improvement, mainly by optimizing the model and training it with more data. A more substantial dataset needs to be found or composed. The architecture of the LSTM RNN can also be updated to gain better results. Even so, our initial goal has not been fully achieved. We are still to train our model for recognizing and separating more instruments. This is a deficient that came mostly from the lack of data available.

Assuming the software produces professional results, a lot of doors open for future developments. This software can be a gateway to other applications such as converting the audio to a midi format, and further transpose it to a music sheet. Another similar network can be taught to recognize chords or beats from the audio. One more software I would personally be interested in is an automatic piano tutorial of the song. After gaining the individual audio of the piano, we could separate the individual notes by analyzing the magnitude spectra and recognize which notes are being played at a given time. 

If further improved, this software will pave the way for numerous MIR algorithms and applications. Assuming that it is optimized to give faultless results, it would be a true landmark for audio signal processing and deep learning in general.


\end{document}
